\documentclass[12pt]{article}
\usepackage{graphicx} % Required for inserting images
\usepackage{amsmath}

\title{Relatório}
\author{Marina Mantovani Meneghel}
\date{February 2024}

\begin{document}

\maketitle

\section{Introduction}
\paragraph{}
Na grande maioria dos fenômenos descritos por equações diferencias parciais (EDPs) não há formas analíticas ou explícitas de se 
conseguir uma solução. Dessa forma, métodos numéricos são necessários para resolver esses problemas.

O Método dos Elementos Finitos é uma poderosa ferramenta utilizada em diversas áreas das ciência computacional, para simulação de 
fenômeos físicos e problemas de engenharia. O método consiste em dividir o domínio, processo conhecido como discretização, em um número 
finito de elementos menores (subdomínios), os quais estão conectados por nós, formando uma malha. Com isso, utilizando formulações 
variacionais das equações governantes que descrevem o problema, constrói-se soluções aproximadas, através da contribuição dos elementos.

Para isso, as equações governantes que descrevem o comportamento de cada elemento são assembladas em uma matriz global, o que permite 
a construção de um sistema linear de equações que será resolvido em todo o domínio.

Uma variação do método tradicional de Elementos Finitos é a aplicação de formulações mistas, onde a pressão e o fluxo são aproximados 
simultanemamente. Essa abordagem é comumente utilizada para descrever a dinâmica dos fluidos. Nela, são consideradas funções pertencentes
ao espaço de aproximação \(H(div)\), que fornece aproximações localmente conservativas. 

A matriz resultante da aplicação desse método é do tipo ponto de sela, a qual é esparsa e não é positiva definida. Resolver sistemas
lineares desse tipo é um problema desafiador, o qual exige técnicas especiais para sua resolução, uma vez que apenas a aplicação de 
métodos tradicionais não são suficientes para a resolução desses problemas.

%Problemas que geram sistema de ponto de sela aparecem em uma variedade de aplicações técnicas e científicas, como na dinâmica dos fluidos, 
%otimização constrained, circutos elétricos e redes, least square estimation, mecânica das estruturas, elementos finitos mistos, até em áreas 
%da economia e de finanças.

\section{Problem}
\paragraph{}
Em geral, os sitema linear pode ser representado da seguinte forma: 

\begin{equation} \label{eq:system}
    \begin{bmatrix}
        A & B \\
        C & D \\
    \end{bmatrix}
    \cdot
    \begin{Bmatrix}
        x \\
        y \\
    \end{Bmatrix}
    =
    \begin{Bmatrix}
        f \\
        g \\
    \end{Bmatrix}
\end{equation}

sendo \(A, B, C\) e \(D\) blocos de matrizes que formam a matriz geral.

Se o sistema descreve um problema de ponto de sela, apresentam um ou mais dos seguintes aspectos: 

\begin{enumerate}
  \item A é simétrica
  \item a parte simétrica de A é positiva semidefinida
  \item B = C 
  \item D é simétrica e positivo semidefinido
  \item C = 0 
\end{enumerate}

Um sistema de ponto de sela, originado de um problema incompressível como de Stokes, Elasticidade ou Darcy, é o caso em que todas 
essas condições são satisfeitas, apresentando a forma abaixo:

\begin{equation} \label{eq:system}
    \begin{bmatrix}
        A & B \\
        B^T & 0 \\
    \end{bmatrix}
    \cdot
    \begin{Bmatrix}
        x \\
        y \\
    \end{Bmatrix}
    =
    \begin{Bmatrix}
        f \\
        g \\
    \end{Bmatrix}
\end{equation}

%Problemos no qual o bloco \(C\) é não nulo aparecem no contexto de método dos elementos finitos misto estabilizado, descretização de equações 
%que descrvem fluidos com certa compressibilidade.

Estamos buscando resolver o problema descrito na Eq. 2 de forma iterativa, introduzindo uma pequena compressibilidade de modo a preencher a 
diagonal nula, resultando na seguinte matriz positiva definida: %como assim positiva definida??



resolvendo um sistema linear menor que o original, sistema reduzido. A principal representação dessa metodologia segregada é a redução pelo complemento de Schur.
 

%Método de Uzawa é bastante popular na mecanica dos fluidos 



%Como a matriz \(A+B (\alpha{}I)^{-1} B^T\) é simétrica positiva definida, o problema é resolvido através da decomposição de Cholesky, para computar 
%a solução inicial. A taxa de convergência aumenta, a medida que \alpha{} diminui. No entanto, o problema torna-se mal condicionado, o número de condicionamento da matriz \(A+B (\alpha{}I)^{-1} B^T\) aumenta com a diminuição de \(\alpha{}\), 
%o que pode levar a problemas de acurácia da solução e de arredondamento.

Sistema de ponto de sela regularizado permite encontrar uma solução inicial estável do problema alterado.

\section{Objectives}
\paragraph{}
A resolução de sistemas lineares do tipo ponto de sela é um problema desafiador, o qual exige técnicas especiais. Portanto, o objetivo
desta pesquisa é a análise da metodologia proposta de resolução de sistemas lineares de equações advindos do Método dos Elementos Finitos, 
e comparação com outras formas de resolver esses problemas. 
%Comparação das técnicas de decomposição direta e iterativa aplicadas a problemas esparsos de ponto de sela, oriundo de aplicações do 
%Método dos Elementos Finitos. 


%biblioteca conta com diversas tecnologias no estado-da-arte do m etodo e a aluna irá utilizá-la para os testes deste projeto de pesquisa.
%NeoPZ

Para isso, serão implementados problemas modelos que gerem sistemas lineares representativos para análise e realização dos testes, a fim de 
analisar as metodologias de resolução desses sistemas. As matrizes serão geradas a partir de problemas de mecânica computacional, variando 
parâmetro de penalização, a dimensão do problemas, ordem de aproximação, refinamento de malha. 

Assim, com essas matrizes, as diferentes metodologias de resolução serão testadas e dados significativos para as análises de performace serão coletados.
Por fim, os dados obtidos serão comparados e conclusões sobre os métodos de resolução de sistema para problemas esparsos de ponto de sela serão 
realizadas.

\section{Methods}
\paragraph{}
Os métodos diretos são responsáveis por fornecer a solução exata, dentro de um número finito de passos, sendo o problema resolvido através 
da decomposição da matriz. Entre as técnicas mais conhecidas, estão a decomposição LU, \(LDL^T\), Cholesky e QR.

Os métodos iterativos fazem uso de sucessivas aproximações, a partir de um chute inicial \(x_0\), para chegar a soluções de maior acurácia, ou seja, 
que se aproximam mais da solução do sistema linear a cada passo (iteração), até atingir uma tolerância de parada estabelecidade. Portanto, não se obtém, 
a partir desses métodos, a resposta exata, mas sim soluções muito próximas do sistema linear, ou seja, que o resíduo das equações se aproxime de zero. 
Os algoritmos mais conhecidos são o Método de Jacobi, Gauss-Seidel, Gradiente Conjugado e GMRES.

A condensação estática é usada para eliminar os graus de liberdade internos de um elemento, reduzindo o número de graus de liberdade do 
sistema global. Consiste, basicamente, de um processo de eliminação de Gauss.
Essa eliminação reduz o tamanho e a complexidade do sistema, o que a torna um eficiente algoritmo numérico para a resolução de sistemas 
lineares de grande porte e esparsos, com a vantagem de que a matriz resultante é simétrica e positiva definida, o que permite a aplicação de 
métodos diretos de resolução como Cholesky. 

\section{Activities}

Inicialmente dedicou-se à ambientação e estudos das ideias principais do Método de Elementos Finitos. Em seguida, um estudo mais aprofundado em 
métodos numéricos para resolução de sistemas lineares foi realizado. 

A partir dos conhecimentos adquiridos, foram executados testes de resolução de sistemas lineares através da linguagem do Mathematica, a partir da 
implementação simples de métodos diretos e iterativos mais usuais, como o Método de Eliminação de Gauss, Método de Jacobi e Gradiente Conjugado. 
Com isso, é possível conhecer como se constrói um algoritmo de resolução de sistemas de equações.

Paralelamente, foi necessário estudo da linguagem C++, com ênfase na programação orientada à objetos, com o objetivo de familiarização 
com a biblioteca NeoPZ, desenvolvida no Laboratório de Mecânica Computacional para cálculos de Elementos Finitos para diversos problemas 
da mecânica computacional. Para testar os conhecimentos, foi construído um programa de integração numérica por meio da Quadratura Gaussiana, 
com o intuito de compreender a estrutura de um programa em C++ e a utilização de classes e métodos. 

Para analisar as metodologias de resolução de sistemas, foi implementado um problema de Darcy, dado da seguinte forma:

%escrever PDE Darcy

gerando a matriz de rigidez. 

A partir disso, foram realizados testes de resolução de sistemas lineares, com a aplicação do Método direto, do método Iterativo não 
condensado e do Iterativo condensado.

Com o objetivo de avaliar a influência do parâmetro de penalização para o método iterativo, experimentações computacionais 
variando o valor de \(\alpha\) e o número de divisões do elementos foram realizados, mantendo fixa a tolerância	como \(10^{-9}\), 
coletando dados como número de iterações, taxa de convergência e tempo de execução.

%Em alguns casos a escolha de alpha leva em conta considerações físicas do problema (pesquisar isso).

%Relação resíduo (log(resíduo)) x número de iterações, para diferentes valores de alpha

Os testes de convergência foram realizados variando o valor de \(\alpha\), e analisando o número de iterações necessárias para a convergência do
problema. O gráfico X mostra a relação entre o número de iterações necessárias para cada \(\alpha\) e a evolução do resíduo, apresentando 
comportamento linear com a norma do resíduo em escala logarítimica. O mesmo foi feito para quantidades maiores de divisões do elemento.


Os resultados mostram que a taxa de convergência aumenta com a diminuição de \(\alpha\), o que é esperado, uma vez que a matriz se aproxima 
da matriz original. Dessa forma, sempre valores descendentes de \(\alpha\) para computar melhores aproximações do vetor solução, com menos 
iterações.

%Relação entre resíduos 

%Comparação Tempo de execução Iterative e Direct, x = divisões e y = tempo, plotando os diversos alphas
O próximo experimento foi realizado para obter os tempos de execução do método iterativo condensado e direto. %Testar mais de uma vez? Da tempos diferentes!!!

Foi possível observar que, dos testes executados, o tempo de execução do método iterativo não foi menor que o do método direto. No entanto, há uma 
tendência de que isso seja superado, a partir de um alpha.

Outro resutado observado a partir desse problema modelo foi de que, no método iterativo condensado, a partir do número de divisões igual a 15, 
o código divergiu para \(\alpha=0.1\). %Por que isso acontece? Quando que alpha=0.01 vai divergir também?

Além disso, houve divergência para \(\alpha<=1.0e-7\). Isso já era esperado para valores de \(\alpha\) pequenos, uma vez que, a medida que 
alpha se aproxima de zero, a matriz torna-se indefinida devido a precisão da máquina, o que impede de usar a fatoração de Cholesky como o 
método direto para resolver o sistema linear. O mesmo não acontece quando a matriz não é condensada, já que utiliza a decomposição LDLt, a 
qual não exige que a matriz seja positiva definida.

\section{Bibliography}

\end{document}
